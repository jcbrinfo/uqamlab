\documentclass[memoire]{UqamLab}
% Indiquez `memoire`, `these` ou `rapport`, selon le cas.
% Pour les document en anglais, ajoutez l’option `english`.

\title{Titre en français}
\titleEnglish{English Title}
\author{Votre Nom}
\matter{informatique}
\matterEnglish{computer science}

% ------------------------------------------------------------------------------

\begin{document}

% ------------------------------------------------------------------------------
% Pages préliminaires
% ------------------------------------------------------------------------------

\include{remerciements}
\tableofcontents
\listoftables
\listoffigures

\begin{resume}
	% Lorsque la langue principale est l’anglais, le résumé en français.
\end{resume}

\begin{abstract}
	% Le résumé, dans la langue principale du document.
\end{abstract}

% ------------------------------------------------------------------------------
% Document principal
% ------------------------------------------------------------------------------

\include{introduction}
…

% ------------------------------------------------------------------------------
% Pages liminaires
% ------------------------------------------------------------------------------

\appendix
\include{appendix1}

\bibliography{votre base de données BibTeX}
\end{document}
